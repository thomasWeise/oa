\hsection{The Termination Criterion}%
\label{sec:terminationCriterion}%
%
We have seen that the search spaces for even small instances of the \gls{JSSP} can already be quite large.
We simply cannot enumerate all points in them, as it would take too long.
While optimization algorithms usually do something much cleverer than simply checking all possible solutions, we already discovered the profoundly simply question:
\inQuotes{If we cannot look at all possible solutions, how can we find the global optimum?}
We then went a step further and asked:
\inQuotes{If we cannot look at all possible solutions, how can we know whether a given candidate solution is the global optimum or not?}
In some optimization scenarios, we can have good lower bounds~\lowerBoundOf{\objf} of the objective function~\objf\ and if we find a solution~$\globalOptimumOf{\solspel}\in\solutionSpace$ with $\objfOf{\globalOptimumOf{\solspel}}\leq\lowerBoundOf{\objf}$, then we know that we have solved the problem.
However, usually the answer to both question is simply:
\emph{We cannot.}

Usually, we simply don't know if the best solution~\bestSoFarOf{\solspel} we know so far is the global optimum or not.
This leads us to another problem:
If we do not know whether we found the best-possible solution or not, how do we know if we can stop the optimization process (whatever that might be) or should continue trying to solve the problem?

At first glance, there are two very simple answers to this question:
We stop either \inQuotes{when the time is up} or \inQuotes{when we found a reasonably-good solution.}
Additionally, there are many more complicated answers~\cite{GCJ2017TCIEAAS}, such as \inQuotes{when we were not able to improve upon~\bestSoFarOf{\solspel} for a while.}%
%
\hsection{Definitions}%
\label{sec:termination:defs}%
%
\begin{definition}[Termination Criterion]%
\label{def:terminationCriterion}%
The \emph{termination criterion}~$\shouldTerminate:\mathSpace{S}\mapsto\{\codeil{False}, \codeil{True}\}$ is a function of the state~\mathSpace{S} of the optimization process which becomes \codeil{True} if the optimization process should stop and remains \codeil{False} as long as it can continue.%
\end{definition}%
%
Into such a termination criterion we can embed any combination of time or solution quality limits.
We could, for instance, define a goal objective value~\goalF\ good enough so that we can stop the optimization procedure as soon as a candidate solution~$\solspel\in\solutionSpace$ has been discovered with $\objfOf{\solspel}\leq\goalF$, i.e., which is at least as good as the goal.
We could set $\goalF=\lowerBoundOf{\objf}$ if we have a reasonably good lower bound.
We could also set to some acceptable quality limit resulting from our practical application scenario.

Alternatively -- or in addition -- we may define a maximum amount of time the user is willing to wait for an answer, i.e., a computational budget after which we simply need to stop.

Most optimization processes iteratively try to find better solutions.
We could stop when, for a certain amount of time, no better solution was discovered.
While the current-best solution may not be \inQuotes{reasonably good,} continuing to try finding better solutions may be pointless anyway when the last improvement was made one minute after we started the optimization process {\dots} and, by now, that is ten hours ago{\dots}

Some algorithms maintain sets of solutions and iteratively work on all the solutions in the sets.
A maximum number of such iterations was used as stopping criterion in some older works, but this is known to be a bad practice and, thus, strongly discouraged~\cite{RLMC2022MNOGAASCCH}.%
%
\endhsection%
%
\hsection{Example: Job Shop Scheduling}%
\label{sec:jssp:termination}%
%
In our example domain, the \gls{JSSP}, we can assume that the human operator will input the instance data~\instance\ into the computer.
Then she may go drink a coffee and expect the results to be ready upon her return.
While she does so, can we solve the problem?
Unfortunately, probably not.
As said, for finding the best possible solution, if we are unlucky, we would need in invest a runtime growing exponentially with the problem size, i.e., \jsspMachines\ and \jsspJobs~\cite{LLRKS1993SASAAC,CPW1998AROMSCAAA}.
So can we guarantee to find a solution which is, say, at most 1\% worse, until she finishes her drink?
Unfortunately, it was shown that there is \emph{no} algorithm which can guarantee us to find a solution at most only 25\%~worse than the optimum within a runtime polynomial in the problem size~\cite{WHHHLSS1997SSS,JMSO2005ASFJSSPWCPT} in \citeyear{WHHHLSS1997SSS}.
Since \citeyear{MS2011HOAFAJSSP}, we know that \emph{any} algorithm guaranteeing to provide schedules that are only be a constant factor (be it 25\% or 1'000'000) worse than the optimum may need the dreaded exponential runtime~\cite{MS2011HOAFAJSSP}.
So whatever algorithm we will develop for the \gls{JSSP}, defining a some limit solution quality based on the lower bound of the objective value at which we can stop as the \emph{only} termination criterion makes little sense, because the runtime needed to get there may simply be too long.

Hence, we let us look at this practically:
The operator enters the problem instance data.
Then she drinks a coffee.
A termination criterion granting \jsspRuntime\ of runtime seems to be reasonable to me here.
We should look for the algorithm implementation that can give us the best solution quality within that time window.

Of course, there may also be other constraints based on the application scenario.
For example, it could be that only Gantt charts that can be implemented/completed within the working hours of a single day are acceptable (\emph{feasible}) solutions in a real-world \gls{JSSP}.
We then might let the algorithm run longer than \jsspRuntime\ until such a solution was discovered.
But, as said before, if an odd scenario occurs, it might take a long time to discover such a solution, if ever.

The human operator may also need to be given the ability to manually stop the process and extract the best-so-far solution~\bestSoFarOf{\solspel} if need be.
For our benchmark instances, however, this is not relevant and we can limit ourselves to the runtime-based termination criterion.

We will investigate many different optimization methods.
Some are designed to make quick improvements and to find \emph{reasonably good} solutions quickly.
Others are designed to perform a broad search, to investigate many possible beneficial solution traits, in the hope to eventually find a \emph{very good} solution.
Now with \jsspRuntime, we set a fairly short runtime limit.
Algorithms which do a broader search will appear to not perform well, because there will not be enough time for their advantages to become visible.
Alternatively, if we had chosen a long runtime limit, then the quickly improving algorithms would appear worse.
We discuss this issue in-depth in TODO.
We will try out all the algorithms that we discuss.
Please keep in mind that the fact that some will work out worse than others in our particular scenario does not mean that they are worse in general.
\endhsection%
%
\hsection{Summary}%
In this section, we raised the question when an optimization should stop.
This is an important question, since we already know that we usually do not know whether we found the optimal solution or not.
Sometimes, termination criteria arise from our practical scenario, e.g., our application defines a time limit and then, that is when we will stop.
Sometimes, we have a lot of time available and could let the optimization process go on for quite a bit.
However, even then, we do not want to waste computational time when it is already clear that our algorithm cannot make any improvement anymore.
Finally, we may have a goal quality threshold~\goalF\ and after reaching it, we can stop.
Sometimes we have a combination of the above.
Either way, we have to stop eventually.%
\endhsection%
\endhsection%
%

