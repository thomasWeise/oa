\hsection{Summary}%
%
The examples we have discussed so far are, actually, related to each other.
They all fit into the broad areas of operational research and smart manufacturing~\cite{DEPBS2012SMMIADDP,HPO2016DPFI4S}.
The goal of \indexi{smart manufacturing} is to optimize development, \indexi{production}, and \indexi{logistics} in the \indexi{industry}.
Therefore, computer control is applied to achieve high levels of adaptability in the multi-phase process of creating a product from raw material.
The manufacturing processes and maybe even whole supply chains are networked.
Product lot sizes are small and a high flexibility is required to adapt production processes to customer wishes.
This creates the requirement for a large degree of automation and automatic intelligent decisions.
The key technology necessary to propose such decisions are optimization algorithms.
In a perfect world, the whole production process as well as the warehousing, packaging, and logistics of final and intermediate products would take place in an \emph{optimized} manner.
No time or resources would be wasted as production gets cleaner, faster, and cheaper while the quality increases.%
%
\endhsection%
%

